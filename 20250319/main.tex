\documentclass[9pt]{beamer}
\usepackage[T1]{fontenc}
\usepackage[english]{babel}
\usepackage{xcolor}
\usepackage{amsfonts}
\usepackage{amsmath}
\usepackage{amssymb}
\usepackage{mathtools}
\usepackage{blkarray}
\usepackage{bm}
\usepackage{physics}
\usepackage{etoolbox}
\usepackage{eqparbox}
\usepackage{graphicx}

% Fonts
\renewcommand{\rmdefault}{ptm}
\renewcommand{\sfdefault}{phv}

% Spacing
\setlength\parindent{0pt}
\setlength\parskip{1ex plus 1ex minus 0.5ex}

% Basic notation
\DeclareMathOperator*{\argmax}{arg\,max}
\DeclareMathOperator*{\argmin}{arg\,min}
\renewcommand\d{\mathop{}\!\textnormal{\slshape d}}
\newcommand\eye{\mathop{}\!\mathbb{I}}
\newcommand\defined{\doteq}
\newcommand\upto{\stackrel{+}{=}}
\newcommand\without{\setminus}
\newcommand{\describe}[3][0pt]{\hspace*{.12em}\underbracket[0.5pt][1pt]{#2\hspace*{#1}}_\text{#3}}
\newcommand{\describel}[2]{\Shortunderstack[l]{$\underbracket[1pt][0pt]{#1}$ \raisebox{.8ex}{\scriptsize\rlap{#2}}}}
\newcommand{\changes}[1]{\underbracket[1pt][0pt]{#1}}

% Equation
\makeatletter
\newcommand{\removeParBefore}{\ifvmode\vspace*{-\baselineskip}\setlength{\parskip}{0ex}\fi}
\newcommand{\removeParAfter}{\@ifnextchar\par\@gobble\relax}
\newcommand{\eq}{\begingroup\removeParBefore\endlinechar=32 \eqinner}
\newcommand{\eqinner}[2][aligned]{\endlinechar=32%
\begin{gather}\begin{#1}#2\end{#1}\end{gather}\endgroup\removeParAfter}
\makeatother

% Density
\DeclareDocumentCommand{\p}{ D<>{p} D<>{} r() }{
\def\content{#3}\patchcmd{\content}{|}{\;#2\vert\;}{}{}
\ensuremath{#1 #2(\content #2)}}

% Probability
\DeclareDocumentCommand{\P}{ D<>{P} D<>{\big} r() }{
\def\content{#3}\patchcmd{\content}{|}{\;#2\vert\;}{}{}
\ensuremath{\operatorname{#1}#2(\content #2)}}

% Expectation
\DeclareDocumentCommand{\E}{ D<>{E} E{_}{{}} D<>{\big} r[] }{
\def\content{#4}\patchcmd{\content}{|}{\;#3\vert\;}{}{}
\ensuremath{\operatorname{#1}_{#2}#3[\content #3]}}

% Divergence
\DeclareDocumentCommand{\D}{ D<>{D} D<>{\big} r[] }{
\def\content{#3}\patchcmd{\content}{||}{\;#2\|\;}{}{}
\ensuremath{\operatorname{#1}\!#2[\content #2]}}

% Distributions
\NewDocumentCommand{\Nor}{ r() }{\P<Normal>](#1)}
\NewDocumentCommand{\Cat}{ r() }{\P<Cat>](#1)}
\NewDocumentCommand{\Bin}{ r() }{\P<Bin>](#1)}
\NewDocumentCommand{\Bet}{ r() }{\P<Beta>](#1)}
\NewDocumentCommand{\Ber}{ r() }{\P<Bernoulli>(#1)}
\NewDocumentCommand{\Dir}{ r() }{\P<Dir>(#1)}

% Information
\DeclareDocumentCommand{\KL}{ D<>{\big} r[] }{\D<KL><#1>[#2]}
\DeclareDocumentCommand{\H}{ D<>{\big} r[] }{\E<H><#1>[#2]}
\DeclareDocumentCommand{\I}{ D<>{\big} r[] }{\E<I><#1>[#2]}

% Symbols
\newcommand{\cmark}{\textcolor{green}{\ding{51}}}
\newcommand{\xmark}{\textcolor{red}{\ding{55}}}

% Shortcuts
\DeclareDocumentCommand{\lnpp}{ D<>{} r() }{
\ensuremath{\p<\ln p_\phi><#1>(#2)}}
\DeclareDocumentCommand{\pp}{ D<>{} r() }{
\ensuremath{\p<p_\phi><#1>(#2)}}
\DeclareDocumentCommand{\qp}{ D<>{} r() }{
\ensuremath{\p<q_\phi><#1>(#2)}}
\DeclareDocumentCommand{\SymLogNormal}{ D<>{} r() }{
\ensuremath{\p<\operatorname{SymLogNormal}><#1>(#2)}}
\newcommand{\sign}{\operatorname{sign}}
% \newcommand{\abs}{\operatorname{abs}}
\newcommand{\eps}{\epsilon}
% \newcommand{\erf}{\operatorname{erf}}
\newcommand{\fot}{\textstyle\frac{1}{2}}
\newcommand{\symlog}{\ensuremath{\operatorname{symlog}}}
\newcommand{\symexp}{\ensuremath{\operatorname{symexp}}}
\newcommand{\twohot}{\ensuremath{\operatorname{twohot}}}
\newcommand{\sg}{\ensuremath{\operatorname{sg}}}
\newcommand{\softmax}{\operatorname{softmax}}
\newcommand{\logsoftmax}{\operatorname{log\,softmax}}
\newcommand{\EMA}{\operatorname{EMA}}
\newcommand{\Per}{\operatorname{Per}}
\newcommand{\ema}{\operatorname{ema}}
\newcommand{\per}{\operatorname{per}}


% Basic packages
\usepackage[utf8]{inputenc}
\usepackage{microtype}
\usepackage{hyperref}
\usepackage{url}
\usepackage{nicefrac}
\usepackage{subcaption}
\usepackage{wrapfig}
\usepackage[ruled, vlined]{algorithm2e}
\usepackage{sidecap}

% TikZ and related libraries
\usepackage{tikz}
\usepackage{pgfplots}
\pgfplotsset{compat=newest}
\usetikzlibrary{arrows,arrows.meta,calc}
\usetikzlibrary{patterns,backgrounds}
\usetikzlibrary{positioning,fit}
\usetikzlibrary{shapes.geometric,shapes.multipart}
\usetikzlibrary{patterns.meta,decorations.pathreplacing,calligraphy}
\usetikzlibrary{tikzmark}
\usetikzlibrary{decorations.pathmorphing}

\usepackage{booktabs}
\usepackage{multirow}
\usepackage{array}
% \usepackage{siunitx}
\usepackage{makecell}

\usepackage[sort]{natbib}

\AtBeginDocument{\RenewCommandCopy\qty\SI}

\pgfplotsset{compat=newest}

\newcommand{\blap}[1]{\vbox to 0pt{\hbox{#1}\vss}}

\newcommand{\enc}{h}
\newcommand{\Infop}{\info_{p,h}}
\newcommand{\Expp}{\Exp_{p,h}}

\newcommand{\x}{\mathbf{x}}
\newcommand{\z}{\mathbf{z}}
\newcommand{\y}{\mathbf{y}}
\newcommand{\w}{\mathbf{w}}
\newcommand{\bo}{\mathbf{o}}
\newcommand{\ba}{\mathbf{a}}
\newcommand{\bz}{\mathbf{z}}
\newcommand{\bs}{\mathbf{s}}
\newcommand{\scoref}{\nabla_\x \log p^\tau(\x)}
\newcommand{\scorem}{\mathbf{S}_\theta(\x, \tau)}
\newcommand{\bbe}{\mathbb{E}}
\newcommand{\Tau}{\mathcal{T}}
\newcommand*{\diris}{$\Delta$-\textsc{iris}}
\newcommand*{\dtoken}{$\Delta$}
\newcommand*{\itoken}{\scalebox{0.9}[0.9]{I}}
% \renewcommand*{\argmin}{argmin}
% \renewcommand*{\argmax}{argmax}
% \renewcommand{\norm}{\big\lVert}{\big\rVert}

\setbeamertemplate{navigation symbols}{}

% Add page number to the lower right corner
\setbeamertemplate{footline}{%
  \hfill\usebeamercolor[fg]{page number in head/foot}%
  \usebeamerfont{page number in head/foot}%
  \insertframenumber\,/\,\inserttotalframenumber\hspace*{1ex}\vskip2pt%
}

% Customize the page number appearance
\setbeamercolor{page number in head/foot}{fg=gray}
\setbeamerfont{page number in head/foot}{size=\small}

\title{Efficient World Models \\ with Context-Aware Tokenization}
\setbeamerfont{title}{size=\huge}
\date{March 12, 2025}

\begin{document}

\begin{frame}
    \titlepage
\end{frame}

\begin{frame}
    \frametitle{Reference}

    \bibliographystyle{plainnat}
    \bibliography{references}
\end{frame}

\begin{frame}
    \frametitle{Background}
    
    \begin{itemize}
        \item \textsc{iris} agent achieved strong results in Atari 100k benchmark
        \begin{itemize}
            \item World model: discrete autoencoder + autoregressive transformer
            \item Dynamics learning as sequence modelling of image tokens
            \item Opened avenues for model-based methods leveraging generative modelling advances
        \end{itemize}
        \vspace{0.5em}
        
        \item Scaling challenges:
        \begin{itemize}
            \item Complex environments require many tokens to encode frames
            \item Sophisticated dynamics need longer memory of past states
            \item Imagination procedure becomes prohibitively slow
            \item Hard to maintain good imagined-to-collected data ratio
        \end{itemize}
        \vspace{0.5em}
        
        \item Approach: \diris (\cite{micheli2024efficientworldmodelscontextaware})
        \begin{itemize}
            \item Attends to trajectory of observations and actions
            \item Encodes stochastic deltas between time steps
            \item Reduces token count and offloads deterministic aspects to autoencoder
            \item Interleaves continuous state summaries with discrete transition tokens
        \end{itemize}
    \end{itemize}
\end{frame}

\begin{frame}
    \frametitle{Differential Tokens}
    
    \begin{itemize}
        \item Key features of \diris:
        \begin{itemize}
            \item Scales to visually complex environments with longer time horizons
            \item Encodes frames by attending to trajectory of observations and actions
            \item Describes stochastic deltas between time steps
        \end{itemize}
        \vspace{0.5em}
        
        \item Benefits of differential encoding:
        \begin{itemize}
            \item Drastically reduces number of tokens needed per frame
            \item Offloads deterministic aspects to autoencoder
            \item Lets transformer focus on stochastic dynamics
        \end{itemize}
        \vspace{0.5em}
        
        \item Challenges and solutions:
        \begin{itemize}
            \item \dtoken-tokens make autoregressive prediction more difficult
            \item Model must integrate over multiple steps to represent current state
            \item Solution: Interleave continuous \itoken-tokens (state summaries) with discrete \dtoken-tokens
        \end{itemize}
    \end{itemize}
\end{frame}

\begin{frame}
    \frametitle{\textsc{iris} vs \diris}
    \begin{center}
    \includegraphics[width=\textwidth]{figures/1.png}
    \end{center}
    \begin{center}
    \vspace{-1em}
    Discrete autoencoder comparison: \textsc{iris} (left) encodes frames independently, requiring $z_t$ to carry all information for reconstruction. \diris\ (right) conditions on past frames/actions, so $z_t$ only captures stochastic changes. This reduces required tokens ($K \ll K_I$), speeding up autoregressive prediction.
    \end{center}
\end{frame}

\begin{frame}
    \frametitle{Background: IRIS}
    
    \begin{itemize}
        \item Discrete autoencoder for image tokenization:
        \begin{itemize}
            \item Encoder maps images to discrete tokens: $E_I: \mathbb{R}^{h \times w \times 3} \rightarrow \{1,\dots,N_I\}^{K_I}$
            \item Decoder reconstructs images from tokens: $D_I: \{1,\dots,N_I\}^{K_I} \rightarrow \mathbb{R}^{h \times w \times 3}$
            \item Trained with reconstruction, perceptual and commitment losses
        \end{itemize}
        \vspace{0.5em}
        
        \item Transformer for dynamics modeling:
        \begin{itemize}
            \item Operates on sequence of image and action tokens
            \item Predicts transitions, rewards, and terminations autoregressively
            \item Trained with cross-entropy on experience segments
        \end{itemize}
        \vspace{0.5em}
        
        \item Key capabilities:
        \begin{itemize}
            \item Builds reusable vocabulary for frame encoding
            \item Attends to history for predictions
            \item Models joint distribution of future states
        \end{itemize}
    \end{itemize}
\end{frame}

\begin{frame}
    \frametitle{Disentangling deterministic and stochastic dynamics - Part 1}
    
    \begin{itemize}
        \item \textsc{iris} limitations:
        \begin{itemize}
            \item Encodes frames independently - no temporal redundancy assumptions
            \item Large token count needed for visually complex frames
            \item Quadratic attention cost limits computation
        \end{itemize}
        \vspace{0.5em}
        
        \item Solution: Condition autoencoder on history
        \begin{itemize}
            \item Only encode changes (deltas) between frames
            \item Deltas often simpler than full frames
            \item Separate deterministic and stochastic components
        \end{itemize}
        \vspace{0.5em}
        
        \item Example: Grid-world movement
        \begin{itemize}
            \item Deterministic: Agent moving based on key press
            \item Stochastic: Random enemy appearances
            \item Only need to encode stochastic events
        \end{itemize}
    \end{itemize}
\end{frame}

\begin{frame}   
    \frametitle{Disentangling deterministic and stochastic dynamics- Part 2}
    
    \begin{itemize}
        \item Set definitions:
        \begin{itemize}
            \item $\mathcal{S}_n (\mathcal{Y}) = \bigcup_{i = 1}^n \mathcal{Y} ^ i$ for tuples up to length $n$
            \item $\mathcal{S} (\mathcal{Y}) = \mathcal{S}_\infty(\mathcal{Y})$ for infinite tuples
            \item Token vocabulary: $\mathcal{Z} = \{1, \dots, N\}$
        \end{itemize}
        \vspace{0.5em}
        
        \item Encoder $E : \mathcal{S} ( \mathcal{X} \times \mathcal{A} ) \times \mathcal{X} \rightarrow \mathcal{Z} ^ K$:
        \begin{itemize}
            \item Input: $(x_0, a_0, \dots, x_{t-1}, a_{t-1}, x_t)$
            \item Output: $z_t = (z_t^1, \dots, z_t^K)$ \dtoken-tokens
            \item CNN-based with vector quantization
        \end{itemize}
        \vspace{0.5em}
        
        \item Decoder $D : \mathcal{S} ( \mathcal{X} \times \mathcal{A} ) \times \mathcal{Z} ^ K \rightarrow \mathcal{X}$:
        \begin{itemize}
            \item Input: $(x_0, a_0, \dots, x_{t-1}, a_{t-1}, z_t)$
            \item Output: Reconstructed frame $\hat{x}_t$
            \item Losses: $L_1 + L_2 + L_\text{max} + L_\text{commit}$
        \end{itemize}
    \end{itemize}
\end{frame}

\begin{frame}
    \frametitle{Modelling Stochastic Dynamics - Part 1}
    
    \begin{itemize}
        \item Challenge: Predicting future \dtoken-tokens is difficult
        \begin{itemize}
            \item Complex integration over past actions and tokens
            \item Example: Random enemy movement in grid world
            \item Hard to predict agent-enemy interactions
        \end{itemize}
        \vspace{0.5em}
        
        \item Solution: Interleaved continuous \itoken-tokens
        \begin{itemize}
            \item Similar to MPEG's \itoken-frames
            \item Create "soft" Markov blanket
            \item Avoid integrating over past \dtoken-tokens
        \end{itemize}
        \vspace{0.5em}
        
        \item \itoken-token Generation:
        \begin{itemize}
            \item Auxiliary CNN processes frames
            \item No discrete autoencoder or reconstruction loss
            \item Optimized end-to-end with dynamics model
        \end{itemize}
    \end{itemize}
\end{frame}

\begin{frame}
    \frametitle{Modelling Stochastic Dynamics - Part 2}
    
    \begin{itemize}
        \item Dynamics Model Input Sequence:
        \begin{itemize}
            \item Past \itoken-tokens $\tilde{x}_i$
            \item Action tokens $a_i$
            \item \dtoken-tokens $z_i^k$
        \end{itemize}
        \vspace{0.5em}
        
        \item Model Predictions:
        \begin{itemize}
            \item Next \dtoken-token distribution: $p_{G}(\hat{z}_{t}^{k+1} | \tilde{x}_{< t}, z_{< t}, a_{< t}, z_{t}^{\le k})$
            \item Reward distribution: $p_{G}(\hat{r}_{t} | \tilde{x}_{\le t} z_{\le t}, a_{\le t})$
            \item Termination distribution: $p_{G}(\hat{d}_{t} | \tilde{x}_{\le t}, z_{\le t}, a_{\le t})$
        \end{itemize}
        \vspace{0.5em}
        
        \item Implementation Details:
        \begin{itemize}
            \item Transformer encoder with causal self-attention
            \item Cross-entropy loss for transitions and terminations
            \item Discrete regression with two-hot targets for rewards
        \end{itemize}
    \end{itemize}
\end{frame}

\begin{frame}
    \frametitle{Modelling Stochastic Dynamics - Figure}
    \begin{center}
    \includegraphics[width=\textwidth]{figures/2.png}
    \end{center}
    \begin{center}
    \vspace{-1em}
    Unrolling dynamics over time. At each step (dashed lines), the GPT-like transformer $G$ predicts \dtoken-tokens for the next frame, plus reward and termination. It takes action tokens, \dtoken-tokens, and \itoken-tokens as input, where \itoken-tokens are continuous embeddings that reduce need to attend to past \dtoken-tokens. Initial frame $x_0$ embeds to \itoken-token $\tilde{x_0}$. From $\tilde{x_0}$ and $a_0$, $G$ predicts reward $\hat{r}_0$, termination $\hat{d}_0$, and autoregressively predicts \dtoken-tokens $\hat{z}_1 = (\hat{z}_1^1, \dots, \hat{z}_1^K)$. During imagination, next frame (stripped box) is computed by decoder $D$ as $x_1 = D(x_0, a_0, \hat{z}_1)$.
    \end{center}
\end{frame}

\begin{frame}
    \frametitle{Policy Improvement}
    
    \begin{itemize}
        \item Learning in imagined POMDP:
        \begin{itemize}
            \item Policy $\pi$ trains using world model $(E,D,G)$
            \item Starts from real observation $x_0$ from experience
            \item Rolls out for $H$ steps or until termination
        \end{itemize}
        \vspace{0.5em}
        
        \item Imagination procedure:
        \begin{itemize}
            \item Policy observes reconstructed state: $\hat{x}_t$
            \item Samples action: $a_t \sim \pi(a_t | \hat{x}_{\le t})$
            \item Model predicts reward $\hat{r}_t$ and termination $\hat{d}_t$
            \item Model generates next tokens: $\hat{z}_{t+1} \sim p_G(\hat{z}_{t+1} | \hat{x}_{\le t}, \hat{a}_{\le t}, \hat{z}_{\le t})$
            \item Decoder reconstructs next observation: $\hat{x}_{t+1} = D(\hat{x}_{\le t}, \hat{a}_{\le t}, \hat{z}_{\le t}, \hat{z}_{t+1})$
        \end{itemize}
        \vspace{0.5em}
        
        \item Training approach:
        \begin{itemize}
            \item Actor-critic method from IRIS
            \item Value baseline predicts $\lambda$-returns
            \item REINFORCE with value baseline
            \item Entropy maximization for exploration
        \end{itemize}
    \end{itemize}
\end{frame}

\begin{frame}
    \frametitle{Experiment}

    \begin{center}
    \scalebox{0.8}{
    \begin{tabular}{lccccc}
    \toprule
    Method                     & Return @1M          & Return @5M           & Return @10M          & \#Parameters   & FPS     \\
    \midrule
    \diris                     & 7.7 (0.5)          & \textbf{15.4} (0.4) & \textbf{16.1} (0.1) & \textbf{25M} & 20          \\
    DreamerV3 XL               & \textbf{9.2} (0.3) & 14.2 (0.2)          & 15.1 (0.3)          & 200M         & \textbf{30} \\
    \textsc{iris} (64 tokens)  & 5.5 (0.7)          & -                    & -                    & 48M          & 2           \\
    \midrule
    \diris\ w/o \itoken-tokens & 6.6 (0.2)          & 10.4 (0.5)          & 12.6 (0.8)          & 24M          & 22          \\
    DreamerV3 M                & 6.2 (0.5)          & 12.6 (0.7)          & 13.7 (0.8)          & 37M          & 40          \\
    \textsc{iris} (16 tokens)  & 4.4 (0.1)          & -                    & -                    & 50M          & 6           \\
    \bottomrule    
    \end{tabular}
    }
    \end{center}

    \begin{center}
    \vspace{-1em}
    Results on Crafter benchmark: \diris\ achieves best performance at 5-10M frames with 8x fewer parameters than DreamerV3 XL, demonstrating strong scaling in a visually complex environment.
    \end{center}
\end{frame}

\begin{frame}
    \frametitle{Experiment}

    \begin{center}
    \includegraphics[width=0.8\textwidth]{figures/3.png}
    \end{center}
    \begin{center}
    \vspace{-1em}
    Removing I-tokens from the input sequence of the autoregressive transformer significantly hurts performance.
    \end{center}
\end{frame}

\begin{frame}
    \frametitle{Reconstruction Errors}

    \begin{center}
        \includegraphics[width=\textwidth]{figures/4.png}
    \end{center}

    \begin{center}
    \vspace{-1em}
    Bottom $1\%$ test frames autoencoded by \diris\ (4 tokens) and \textsc{iris} (16 tokens). Each token takes a value in $\{1, 2, \dots, 1023, 1024\}$, i.e. \diris\ encodes frames with $4 \times \log_{2}(1024) = 40$ bits while \textsc{iris} uses 160 bits. Original frames, reconstructions, and errors are respectively displayed in the top, middle, and bottom rows. Even in the worst instances, \diris\ makes only minor errors, whereas \textsc{iris} fails to accurately reconstruct frames. These errors severely hamper the agent's performance, as it purely learns behaviours from frames generated by its autoencoder.
    \end{center}
\end{frame}

\begin{frame}
    \frametitle{\itoken-tokens}

    \begin{center}
        \includegraphics[width=\textwidth]{figures/5.png}
    \end{center}

    \begin{center}
    \vspace{-1em}
    Trajectories imagined with (top) and without (bottom) \itoken-tokens. The top trajectory shows 30+ seconds of coherent gameplay with complex mechanics learned by \diris' world model. Without \itoken-tokens, the model fails to predict future \dtoken-tokens accurately, leading to glitches that hinder policy learning in an unrealistic environment.
    \end{center}
\end{frame}

\begin{frame}
    \frametitle{Returns on Atari 100k}

    \begin{table}[h]
    \begin{center}
    \centering
    \resizebox{0.9\textwidth}{!}{
    \begin{tabular}{lrr rrrrr}
    \toprule
    Game                 &  Random    &  Human    &  SimPLe    &  DreamerV3      &  \textsc{storm}  &  \textsc{iris}    &  \diris\ (ours)     \\
    \midrule
    Alien                &  228       &  7128     &  617       &  959            &  \textbf{984}    &  420              &  391                \\
    Amidar               &  6         &  1720     &  74        &  139            &  \textbf{205}    &  143              &  64                 \\
    Assault              &  222       &  742      &  527       &  706            &  801             &  \textbf{1524}    &  1123               \\
    Asterix              &  210       &  8503     &  1128      &  932            &  1028            &  854              &  \textbf{2492}                 \\
    BankHeist            &  14        &  753      &  34        &  649            &  641             &  53               &  \textbf{1148}                 \\
    BattleZone           &  2360      &  37188    &  4031      &  12250          &  \textbf{13540}  &  13074            &  11825               \\
    Boxing               &  0         &  12       &  8         &  78             &  \textbf{80}     &  70               &  70                  \\
    Breakout             &  2         &  31       &  16        &  31             &  16              &  84               &  \textbf{302}       \\
    ChopperCommand       &  811       &  7388     &  979       &  420            &  \textbf{1888}   &  1565             &  1183                 \\
    CrazyClimber         &  10781     &  35829    &  62584     &  \textbf{97190} &  66776           &  59324            &  57864                 \\
    DemonAttack          &  152       &  1971     &  208       &  303            &  165             &  \textbf{2034}    &  533                 \\
    Freeway              &  0         &  30       &  17        &  0              &  \textbf{34}     &  31               &  31        \\
    Frostbite           &  65        &  4335     &  237       &  909            &  \textbf{1316}   &  259              &  279                 \\
    Gopher               &  258       &  2413     &  597       &  3730           &  \textbf{8240}   &  2236             &  6445     \\
    Hero                 &  1027      &  30826    &  2657      &  \textbf{11161} &  11044           &  7037             &  7049                 \\
    Jamesbond            &  29        &  303      &  101       &  445            &  \textbf{509}    &  463              &  309                \\
    Kangaroo             &  52        &  3035     &  51        &  4098           &  \textbf{4208}   &  838              &  2269                 \\
    Krull                &  1598      &  2666     &  2205      &  7782           &  \textbf{8413}   &  6616             &  5978               \\
    KungFuMaster         &  259       &  22736    &  14863     &  21420          &  \textbf{26182}  &  21760            &  21534              \\
    MsPacman             &  307       &  6952     &  1480      &  1327           &  \textbf{2674}   &  999              &  1067                 \\
    Pong                 &  -21       &  15       &  13        &  18             &  11              &  15               &  \textbf{20}        \\
    PrivateEye           &  25        &  69571    &  35        &  882            &  \textbf{7781}   &  100              &  103                 \\
    Qbert                &  164       &  13455    &  1289      &  3405           &  \textbf{4523}   &  746              &  1444                 \\
    RoadRunner           &  12        &  7845     &  5641      &  15565          &  \textbf{17564}  &  9615             &  10414              \\
    Seaquest             &  68        &  42055    &  683       &  618            &  525             &  661              &  \textbf{827}                 \\
    UpNDown              &  533       &  11693    &  3350      &  \textbf{9234}  &  7985            &  3546             &  4072                 \\
    \midrule
    \#Superhuman         &  0         &  N/A      &  1         &  9              &  10              &  10               &  \textbf{11}        \\
    Mean                 &  0.00      &  1.00     &  0.33      &  1.10           &  1.27            &  1.05             &  \textbf{1.39}      \\
    Interquartile Mean   &  0.00      &  1.00     &  0.13      &  0.50           &  0.64            &  0.50             &  \textbf{0.65}               \\
    \bottomrule
    \end{tabular}
    }
    \end{center}
    \end{table}
\end{frame}

\end{document}